\documentclass{beamer}

\mode<presentation>
{
	\usetheme{metropolis}
}

\usepackage[english]{babel}

\usepackage[utf8]{inputenc}

\usepackage{times}
\usepackage[T1]{fontenc}

\usepackage{amssymb, amsmath, amscd}
\usepackage{mathrsfs, amsfonts}

\usepackage[all, pdf]{xy}
%\usepackage{cite}
\usecolortheme{rose}
%\usepackage{color}
%\usepackage[usenames,dvipsnames]{color}


\title[Short Paper Title]
{A Bird's Eye View\\on Discrete Logarithm Problems (DLP)\\over Finite Fields}

\author[He]
{Xiao'ou He}

\institute[Universities of Somewhere and Elsewhere]
{Key Laboratory of Mathematics Mechanization, AMSS}

\date[Annual 2017]
{December 14, 2017}

\AtBeginSubsection[]
{
  \begin{frame}<beamer>{Outline}
    \tableofcontents[currentsection,currentsubsection]
  \end{frame}
}

\begin{document}

\bibliographystyle{abbrv}
%\bibliography{dlp v20171212}

\begin{frame}
  \titlepage
\end{frame}

\begin{frame}{Outline}
  \tableofcontents[pausesections]
\end{frame}

%%% =================================================================================== %%%
\section{Motivation}

	\subsection{DLP in Cryptography}
		\begin{frame}{DLP in Cryptography}
			\begin{itemize}
				\item
					Key-exchange scheme [Diffie and Hellman, 1976]
					\pause
				\item
					Encryption algorithm [ElGamal, 1985]
					\pause
				\item
					Digital Signature Algorithm (DSA) [NIST, 1991]
			\end{itemize}
		\end{frame}
%%% ----------------------------------------------------------------------------------- %%%
	\subsection{Exposition of The Problem}
		% I suppose that the vocabulary "exposition" seems more high-end than the vougor "description"
		\begin{frame}{Case in General}
			\begin{itemize}
				\item
					Cyclic group: $(G, \cdot) = \langle g\rangle$
					\pause
				\item
					Based on number-theoretical hard problem
					\pause
					\begin{itemize}
						\item
							$\mathbb{Z} \rightarrow G, x \mapsto h = g^x$
						\item
							$G \rightarrow \mathbb{Z}, h \mapsto x = \log_g{h}$
					\end{itemize}
					\pause
				\item
					Generic Algorithms - $O(\exp(\log|G|))$\\
					\begin{itemize}
						\item
							Pohlig-Hellman
						\item
							Collision Making\\
							Shanks' baby-step giant-step, Pollard's $\rho$, etc.
					\end{itemize}
					\pause
				\item
					Concerning with two class of groups
						\begin{enumerate}
							\item
								Elliptic curves
							\item
								\alert{Multiplicative groups of finite fields}
						\end{enumerate}
			\end{itemize}
		\end{frame}
		
		\begin{frame}{Case over Finite Fields}
			\begin{itemize}
				\item
					Given: $Q$, $g$ and $h$, where $\langle g\rangle = \mathbb{F}_Q^\times$ containing $h$
				\item
					Find: $\log_gh$
			\end{itemize}
		\end{frame}
%%% =================================================================================== %%%
\section{Previous Work}
%%% ----------------------------------------------------------------------------------- %%%
	\subsection{Preliminaries}
		\begin{frame}{Complexity}
			\begin{block}{The $\mathcal{L}$ Notation}
				$$\mathcal{L}_Q(\alpha) = \exp(O((\log Q)^\alpha (\log\log Q)^{1-\alpha}))$$
				\begin{itemize}
					\item 
						$0 \le \alpha \le 1$
						\pause
					\item
						$\mathcal{L}_Q(\alert{0}) = (\log Q)^{O(1)} = \text{poly}(\log Q$),\\
						$\mathcal{L}_Q(\alert{1}) = (\exp(\log Q))^{O(1)} = \text{exp}(\log Q)$
						\pause
					\item
						When $0<\alpha<1$, $\mathcal{L}_Q(\alpha)$ is \textbf{sub-exponential}
				\end{itemize}
			\end{block}
			\pause
			\begin{block}{Quasi-poly}
				$(\log Q)^{O(\log\log Q)}$ is \textbf{quasi-polynomial}, 
				which is smaller than any $\mathcal{L}_Q(\alpha)$ for $\alpha >0$.
			\end{block}
		\end{frame}
%%% ----------------------------------------------------------------------------------- %%%	
	\subsection{From Sub-exp to Quasi-poly}		
		\begin{frame}{Overview}
			\begin{block}{$\mathcal{L}_Q(\alert{\frac{1}{2}})$ \textbf{Index Calculus Method}}
				[Pohlig, 1977] and [Adleman, 1979] independently
			\end{block}
			\pause
			\begin{block}{$\mathcal{L}_Q(\alert{\frac{1}{3}})$ by [\textbf{Coppersmith, 1984]}}
				\begin{itemize}
						\item
							Originally $\mathbb{F}_{2^n}$
						\item
							\textbf{Number Field Sieve} by [Gordon, 1993]
						\item
							\textbf{Function Field Sieve} by [Adleman and Huang, 1999]
				\end{itemize}
			\end{block}
			\pause
			\begin{block}{$\mathcal{L}_Q(\alert{\frac{1}{4}})$ by [Joux, 2013] to \alert{quasi-poly} by [BGJT@EC'14]}
				\begin{itemize}
					\item 
						Originated from [Joux and Lercier@EC'06]
					\item
						For small char., roughly $p \leq \mathcal{L}_Q(\frac{1}{3})$.
					\item
						Heuristics
				\end{itemize}
			\end{block}
		\end{frame}
%%%%%%%%%%%%%%%%%%%%%%%%%%%%%%%%%%%%%%%%%%%%%%%%%%%%%%%%%%%%%%%%%%%%%%%%%%%%%%%
		\begin{frame}{Smooth - Set A Bound}
			\begin{definition}
				Given $B>0$. 
				$\forall n\in\mathbb{Z}$ is called \alert{$B$-smooth} if all its prime factors are no larger than $B$. 
				Thus denote \alert{factor basis} as
				$$\mathcal{F}(B) = \{ p\in\mathbb{N}: \text{prime and } p\leq B\}$$
			\end{definition}
			\pause
			\begin{definition}
				Given $B\in\mathbb{Z}_{>0}$.
				$\forall f[X]\in\mathbb{F}_q[X]$ is called \alert{$B$-smooth} if all its irreducible factors are of degree no higher than $B$.
				Thus denote respective \alert{factor basis} as
				$$\mathcal{F}_q(B) = \{F[X]\in\mathbb{F}_q[X]: \text{irr. monic and of }\deg\leq B\}$$
			\end{definition}
		\end{frame}
%%%%%%%%%%%%%%%%%%%%%%%%%%%%%%%%%%%%%%%%%%%%%%%%%%%%%%%%%%%%%%%%%%%%%%%%%%%%%%%
		\begin{frame}{Index Calculus Method - Framework of Following Algorithms}
			\begin{itemize}
				\item Given: $h \in  \mathbb{F}_Q^\times = \langle g\rangle$; Find: $\log_g h$
					\pause
				\item Process:
					\begin{enumerate}
						\item\textbf{Main Phase}
						\begin{itemize}
							\item<3-> \alert{Initialization} Fix parameter $B$, thus $\mathcal{F}(B)$ is also given.
							\item<4-> \alert{Smoothness Selection} Random $c\in[1, Q - 2]$ s.t. $g^c$ is $B$-smooth:
								$$g^c = \prod_{p\in\mathcal{F}(B)} p^{v(p, c)}$$
							\item<5-> \alert{Relation Collection} 
								Take $\log_g$ on both sides and substitute $\log_g p$ by unknown variable $x_p$ 
								(denoted as $\log_g p \leftarrow x_p$):
								$$c \equiv \sum_{p\in\mathcal{F}(B)} v(p, c) x_p \mod Q - 1$$
							\item<6-> \alert{Linea Algebra} Solve system of linear equations.
						\end{itemize}
						\item\textbf{Individual Logarithm}\\
						\uncover<7->{
						Find $b\in[1, Q-2]$ s.t. $g^b h$ is $B$-smooth. 
						\uncover<8->{Then factorization $g^b h = \prod p^{v_0(p)}$ implies
						$$\log_g h \equiv \sum v_0(p) \log_g p - b \mod Q - 1$$.}}
					\end{enumerate}
			\end{itemize}
		\end{frame}

%		\begin{frame}{Index Calculus Method - Review}
%			\begin{block}{Given: $h \in  \mathbb{F}_Q^\times = \langle g\rangle$}
%				To find $\log_g h$, it suffices to find $\log_g F$ for $F\in\mathcal{F}$.\\
%				\pause
%				With $\prod_{F\in\mathcal{F}} F^{v(F)} = 1$, \pause we have
%					$$\sum_{F\in\mathcal{F}} v(F) \log_g F \equiv 0 \mod Q-1$$
%				and treat it as linear equation $\sum_{F\in\mathcal{F}} v(F) x_F \equiv 0 \mod Q-1$.
%			\end{block}
%		\end{frame}
%%%%%%%%%%%%%%%%%%%%%%%%%%%%%%%%%%%%%%%%%%%%%%%%%%%%%%%%%%%%%%%%%%%%%%%%%%%%%%%
		\begin{frame}{Coppersmith's Method - $\mathbb{F}_Q = \mathbb{F}_q[X]/\langle I_k(X)\rangle = \mathbb{F}(\alpha)$}
			\begin{itemize}
				\item Given: $h \in \mathbb{F}_Q^\times = \langle g\rangle$ where $Q = p^k$; Find: $\log_g h$.
					\pause
				\item Process:
					\begin{enumerate}
						\item \textbf{Main Phase}
							\begin{itemize}
								\item<3-> 
									\alert{Initialization} 
									Fix $B$ then obtain $\mathcal{F}_q(B)$.\\
									Find $n$ satisfying $p^{n-1} < k \leq p^n$, fix $n_1 + n_2 = n$.\\
									\uncover<4->{
										Find $S(X)$ of $\deg\leq B$ s.t. \textcolor{mLightGreen}{$\exists I_k(X)|X^{p^n} - S(X)$} with a root $\alpha$.}
								\item<5-> \alert{Smoothness Selection} 
									Random $G_1(X)$, $G_2(X)$ of $\deg\leq B$ s.t. 
									both $C(X) = G_1(X) + X^{p^{n_1}}G_2(X)$ and $D(X) = C(X)^{p^{n_2}}$ %= A(X^{q^{n_2}}) + S(X)B(X^{q^{n_2}})$ 
									are $B$-smooth, notice $\deg\leq p^{n_1} + B$ and $(p^{n_2} + 1)B$ respectively.
								\item<6-> \alert{Relation Collection} Known
									$$(\prod F(X)^{v(C, F)})^{p^{n_2}} \equiv \prod F(X)^{v(D, F)} \mod I_k(X)$$
									$X \leftarrow \alpha$, take $\log_g$ on both sides, and $\log_g F(\alpha)\leftarrow x_F$
									$$\sum(p^{n_2} v(C, F) - v(D, F)) x_F \equiv 0 \mod Q - 1$$
								\item<7-> \alert{Linear Algebra} Solve system of linear equations.
							\end{itemize}
						\item \textbf{Individual Logarithm Phase} \uncover<8->{Descent strategy}
					\end{enumerate}
			\end{itemize}
		\end{frame}
%%%%%%%%%%%%%%%%%%%%%%%%%%%%%%%%%%%%%%%%%%%%%%%%%%%%%%%%%%%%%%%%%%%%%%%%%%%%%%%
		\begin{frame}{Joux\&Lercier@EC'06}
			\begin{block}{Given: $h\in\mathbb{F}_Q^\times =\langle g\rangle$ where $Q = q^k$}
			Find $S_1(X)$, $S_2(X)\in\mathbb{F}_q[X]$ s.t. $\exists \textcolor{mLightGreen}{I_k(X)|X - S_1(S_2(X))}$, a root $\alpha$.
			Let $\beta = S_2(\alpha)$, then $\exists\textcolor{mLightGreen}{J_k(Y)|Y-S_2(S_1(Y))}$ with root $\beta$.
			\pause
			$$\xymatrix{
				& \mathbb{F}_q[X, Y] \ar[dl]_{\mod Y - S_2(X)} \ar[dr]^{\mod X - S_1(Y)} & \\
				\mathbb{F}_q[X] \ar[dr]_{\mod X - \alpha} & & \mathbb{F}_q[Y] \ar[dl]^{\mod Y - \beta} \\
				& \mathbb{F}_Q & 
			}$$
			\pause
			Start with $G_1(Y)X+G_2(Y)$, we will reach the relation
			$$G_1(S_2(\alpha))\alpha+G_2(S_1(\alpha)) = G_1(\beta)S_1(\beta)+G_2(\beta)$$
			\end{block}
		\end{frame}
%%%%%%%%%%%%%%%%%%%%%%%%%%%%%%%%%%%%%%%%%%%%%%%%%%%%%%%%%%%%%%%%%%%%%%%%%%%%%%%		
		\begin{frame}{Joux\&Lercier@EC'06}
			\begin{itemize}
				\item
					Given: $h\in\mathbb{F}_Q^\times = \langle g\rangle$ where $Q = q^k$; Find: $\log_g h$
				\item
					Process:
					\begin{enumerate}
						\item
							\textbf{Main Phase}
						\begin{itemize}
							\item
								\alert{Initialization} 
								Fix parameter $B$ then obtain $\mathcal{F}_q(B)$. 
								Find $S_1(X)$, $S_2(X)\in\mathbb{F}_q[X]$ of $\deg$ $d_1$, $d_2$ 
								satisfying $d_1d_2\geq k$ s.t. 
								$\exists \textcolor{mLightGreen}{I_k(X)|X - S_1(S_2(X))}$ 
								with a root $\alpha$
							\item
								\alert{Smoothness Selection} 
								Random $G_1(X)$, $G_2(X)$ of $\deg\leq B$ s.t. 
								both $G_1(S_2(X))X+G_2(S_1(X))$ and $G_1(X)S_1(X)+G_2(X)$ are $B$-smooth, 
								notice $\deg\leq B d_2 + 1$, $B d_1$ respectively.
							\item
								\alert{Relation Collection} Known
								$G_1(S_2(\alpha))\alpha+G_2(S_1(\alpha)) = G_1(\beta)S_1(\beta)+G_2(\beta)$
								and smoothness implies 
								$\prod F(\alpha)^{v(F)} = \prod F(\beta)^{w(F)}$.
								Then take $\log_g$ and $F(\alpha)\leftarrow x_F$, $F(\beta)\leftarrow y_F$
								$$\sum v(F) x_F \equiv \sum w(F) y_F \mod Q-1$$
							\item
								\alert{Linear Algebra} Solve system of linear equations.
						\end{itemize}
						\item
							\textbf{Individual Logarithm Phase} Descent strategy.
					\end{enumerate}
			\end{itemize}
		\end{frame}
%%%%%%%%%%%%%%%%%%%%%%%%%%%%%%%%%%%%%%%%%%%%%%%%%%%%%%%%%%%%%%%%%%%%%%%%%%%%%%%		
		\begin{frame}{[Joux, 2013] and [BGJT@EC'14] - Half Relation}
			\begin{block}{Given: $h\in\mathbb{F}_Q^\times =\langle g\rangle$ where $Q = q^k$ and $q>k$}
				Take advantage of the itentity
				$X^q - X = \prod_{\gamma\in\mathbb{F}_q} (X - \gamma)$.\\
				\pause
				Find $S(X)\in\mathbb{F}_q$ s.t. $\exists\textcolor{mLightGreen}{I_k(X)|X^q - S(X)}$ with a root $\alpha$.
				$$\xymatrix{
					& \mathbb{F}_q[X, Y] \ar[dl]_{\mod Y - X^q} \ar[dr]^{\mod Y - S(X)} & \\
					\mathbb{F}_q[X] \ar[dr]_{\mod X - \alpha} & & \mathbb{F}_q[Y] \ar[dl]^{\mod X - \alpha}\\
					& \mathbb{F}_Q &
				}$$
				\pause
				From $G_1(Y)G_2(X)-G_1(X)G_2(Y)$ we can reach the relation
				$$\prod_{\gamma\in\mathbb{F}_q}(G_1(\alpha)-\gamma G_2(\alpha)) = G_1(S(\alpha))G_2(\alpha) - G_1(\alpha)G_2(S(\alpha))$$
			\end{block}
		\end{frame}

		\begin{frame}{[Joux, 2013] and [BGJT@EC'14]}
			\begin{itemize}
				\item
					Given: $h\in\mathbb{F}_Q^\times =\langle g\rangle$ where $Q = q^k$ and $q>k$; Find: $\log_g h$\\
				\item
					Process
					\begin{enumerate}
						\item\textbf{Main Phase}
						\begin{itemize}
							\item
								\alert{Initialization} 
									Fix parameter $B$ then we obtain $\mathcal{F}_q(B)$
									Find $S(X)\in\mathbb{F}_q$ of $\deg\leq B$ s.t. 
									$\exists\textcolor{mLightGreen}{I_k(X)|X^q - S(X)}$ with a root $\alpha$.
							\item
								\alert{Smoothness Selection} 
								Random $G_1(X)$, $G_2(X)$ of $\deg\leq B$ s.t. 
								$G_1(S_2(X))G_2(X) - G_1(X)G_2(S(X))$ is $B$-smooth, 
								notice $\deg\leq B(B+1)$.
							\item
								\alert{Relation Collection} Known
								$\prod_{\gamma\in\mathbb{F}_q}(G_1(\alpha)-\gamma G_2(\alpha)) = G_1(S(\alpha))G_2(\alpha) - G_1(\alpha)G_2(S(\alpha))$
								and smoothness implies 
								$\prod (G_1(\alpha)-\gamma G_2(\alpha)) = \prod F(\alpha)^{v(F)}$.
								Then take $\log_g$ and $G_1(\alpha)-\gamma G_2(\alpha)\leftarrow x_\gamma$, $F(\alpha)\leftarrow x_F$
								$$\sum x_\gamma \equiv \sum v(F) x_F \mod Q-1$$
							\item
								\alert{Linear Algebra} Solve system of linear equations.
						\end{itemize}
						\item\textbf{Individual Logarithm Phase} Descent strategy.
					\end{enumerate}
			\end{itemize}
		\end{frame}
%%% ----------------------------------------------------------------------------------- %%%
	\subsection{Summary}
		\begin{frame}{Paradigm with Transitions}
			\begin{enumerate}
				\item \textbf{Main Phase}
				\begin{itemize}
					\item 
						\alert{Initiation} 
							\uncover<2->{Field extension brings freedom in presentation}
					\item 
						\alert{Smoothness Selection}
							\uncover<3->{Randomly choose $\Rightarrow$ Sieve}
					\item
						\alert{Relation Collection}\\
							\uncover<4->{Understanding (virtual) logarithms}
					\item \alert{Linear Algebra} 
							\uncover<5->{Take advantage of the sparseness}
				\end{itemize}
				\item \textbf{Individual Logarithm Phase} \uncover<6->{Descent Strategy}
			\end{enumerate}
		\end{frame}
%%% =================================================================================== %%%
\section{Research Plan}
	\begin{frame}{Preparations}
		\begin{itemize}
			\item
				\textbf{Algorithm}\\
				\uncover<2->{\emph{Algorithmic Cryptanalysis} by Joux\\
				Course taken: Computational Number Theory}
			\item
				\textbf{Number theory}\\
				\uncover<2->{\emph{A Classical Introduction to Modern Number Theory}\\
				by Kenneth, Ireland and Rosen}
			\item
				\textbf{Polynomials over finite fields}
				\uncover<2->{2017 summer schools}
			\item
				\textbf{Lattice}
				\uncover<2->{Lectures}
			\item
				\textbf{Algebraic geometry}
				\uncover<2->{Course taken}				
		\end{itemize}
	\end{frame}

	\begin{frame}{Research Plan}
		\begin{itemize}
			\item \textbf{Heuristics}: From assumption to rigorous.\\
				CWZ, 2014: Three Heuristics
				\pause
			\item \textbf{Relation obtaining}\\
				Use other identities in finite fields and number theory.
				\pause
			\item \textbf{As for other characteristics}\\
				Generalization of FFS and NFS.
		\end{itemize}
	\end{frame}
	
	\begin{frame}
		Thank you! Any questions?
	\end{frame}
	
	\begin{frame}{References}
		\tiny{\begin{enumerate}
			\item
				Diffie W., Hellman M.E., \emph{New directions in cryptography},
				IEEE Trans. Inf. Theory, \textbf{22}(6), 644-654(1976). 
			\item
				Adleman L., \emph{A subexponential algorithm for the discrete logarithm problem with applications to cryptography},
				In: Proceedings of the 20th Annual Syposium on Foundations of Computer Science: FOCS'79, 55-60.
			\item
				Coppersmith D., \emph{Fast evaluation of logarithms in fields of characteristic two},
				IEEE Trans. Inf. Theory, \textbf{30}(4): 587-593(1984).
			\item
				ElGmal T., \emph{A publick key cryptosystem and a signature scheme based on discrete logarithms},
				IEEE Trans. Inf. Theory, \textbf{31}(4): 469-472(1985).
			\item
				Adleman L., Huang M.D., \emph{Function field sieve method for discrete logarithms over finite fields},
				Inf. Comput., \textbf{151}(1-2): 5-16(1999).
			\item
				Joux A., Lercier R., \emph{The function field sieve in the medium prime case},
				In:Advances in Cryptography: EUROCRYPT'2006, 254-270.
			\item
				Joux A., \emph{A new index calculus algorithm with complexity $\mathcal{L}(1/4 + o(1))$ in small characteristic},
				In: Selected Areas in Cryptography: SAC'2013, 355-379.
			\item
				
			\item
				Barculescu R., Gaudry P., Joux A., Thom$\acute{\text{e}}$ E., \emph{A heuristic quasi-polynomial algorithm for discrete logarithm in finite fields of small characteristic}, In: Advances in Cryptography: EUROCRYPT'2014, 1-16.
			\item
				Joux A., Pierrot C., \emph{Technical history of discrete logarithms in small characteristic finite fields}, 
				Des. Codes Cryptogr. \textbf{78}: 73-85(2016).
			\end{enumerate}
			}
	\end{frame}
\end{document}
