\documentclass{beamer}

\mode<presentation>
{
	\usetheme{metropolis}
}

\usepackage[english]{babel}

\usepackage[utf8]{inputenc}

\usepackage{times}
\usepackage[T1]{fontenc}

\usepackage{amssymb, amsmath, amscd}
\usepackage{mathrsfs, amsfonts}

%\usepackage{color}
%\usepackage[usenames,dvipsnames]{color}

\title[Short Paper Title]
{A Bird's Eye View\\on Discrete Logarithm Problems\\over Finite Fields}

\author[He]
{Xiao'ou He}

\institute[Universities of Somewhere and Elsewhere]
{Key Laboratory of Mathematics Mechanization, AMSS}

\date[Annual 2017]
{December 14, 2017}


\begin{document}

\begin{frame}
  \titlepage
\end{frame}

\begin{frame}{Outline}
  \tableofcontents[pausesections]
\end{frame}

%%% =================================================================================== %%%
\section{Motivation}

	\subsection{Origin of The Problem in Cryptography}
		\begin{frame}{Public-Key Cryptography}
			\begin{itemize}
				\item
					Proposed by Diffie and Hellman, 1976
				\item
					Based on number-theoretical hard problems
					\begin{itemize}
						\item
							Integer Factorizations
						\item
							\alert{Discrete Logarithms}\\
							Concerning with two class of groups:
							\begin{enumerate}
								\item
									Elliptic curves
								\item
									\alert{Multiplicative groups of finite fields}
							\end{enumerate}
					\end{itemize}
			\end{itemize}
		\end{frame}
%%% ----------------------------------------------------------------------------------- %%%
	\subsection{Exposition of The Problem}
		% I suppose that the vocabulary "exposition" seems more high-end than the vougor "description"
		\begin{frame}{Case in General}
			\begin{itemize}
				\item
					Cyclic group: $(G, \cdot) = <g>$.
				\item
					Exponential v. Logarithm
					\begin{itemize}
						\item
							$x \mapsto h = g^x$
						\item
							$h \mapsto x = \log_g{h}$
					\end{itemize}
				\item
					Generic Algorithms:\\
					\begin{itemize}
						\item
							Pohlig-Hellman: Two reductions - CRT and $p^e$ to $p$
						\item
							Collision Making: Baby-step giant-step, Pollard's rho Method, etc.
					\end{itemize}
			\end{itemize}
		\end{frame}
		
		\begin{frame}{Case over Finite Fields}
			\begin{itemize}
				\item
					Given: $Q$, $g$ and $h$, where $<g> = \mathbb{F}_Q^\times$ containing $h$.
				\item
					Find: $\log_gh$.
			\end{itemize}
		\end{frame}
%%% =================================================================================== %%%
\section{Previous Work}
%%% ----------------------------------------------------------------------------------- %%%
	\subsection{Preliminaries}
		\begin{frame}{The $\mathcal{L}$ Notation: Complexity}
			$$\mathcal{L}_Q(\beta, c) = \exp((c + o(1))(\log Q)^\beta (\log\log Q)^{1-\beta})$$
			\begin{itemize}
				\item 
					$c>0$ and $0 \le \beta \le 1$
				\item
					$\mathcal{L}_Q(0, c) = (\log Q)^{c + o(1)} = \text{poly}(\log Q$),\\
					$\mathcal{L}_Q(1, c) = (\exp(\log Q))^{c+o(1)} = \text{exp}(\log Q)$. 
				\item
					When $0<\beta<1$, $\mathcal{L}_Q(\beta, c)$ is \textbf{sub-exponential}.
			\end{itemize}
		\end{frame}
%%% ----------------------------------------------------------------------------------- %%%	
	\subsection{Overview}		
		\begin{frame}{Overview}
			\begin{itemize}
				\item
					\textbf{Index Calculus Method}
						\begin{itemize}
							\item
								1st sub-exp algo., complexity $\mathcal{L}_Q(\alert{\frac{1}{2}}, \cdot)$
							\item
								Adleman, 1979 and Pohlig, 1977 independently
						\end{itemize}
				\item
					\textbf{Coppersmith's Method, 1984}
						\begin{itemize}
							\item
								1st algo. of complexity $\mathcal{L}_Q(\alert{\frac{1}{3}}, \cdot)$.
							\item
								Originally $\mathbb{F}_{2^n}$, then generalized to prime powers (easily).
						\end{itemize}
				\item
					\textbf{Function Field Sieve, 1999}
						\begin{itemize}
							\item
								More general: used for several record breaking comutations.
							\item
								Adleman and Huang
						\end{itemize}
				\item
					\textbf{Number Field Sieve}, 
			\end{itemize}
		\end{frame}
		
		\begin{frame}{Overview: Until 2013}
			All cases solved in $\mathcal{L}_Q(\frac{1}{3}, \cdot)$
			\begin{itemize}
				\item
					Small char.: function field sieve (FFS)
				\item	
					Medium and large char.: number field sieve (NFS)
				\item
					Measure: solve $p = \mathcal{L}_Q(\beta, c)$ where $Q = p^n$,\\
					for relation between $\beta$ and $\frac{1}{3}$, $\frac{2}{3}$ % explain in detail
			\end{itemize}
		\end{frame}

		\begin{frame}{Overview: Small Char.}
			\begin{itemize}
				\item
					1st algo. of complexity $\mathcal{L}_Q(\alert{\frac{1}{4}}, \cdot)$:\\
					Frobenius representation, Joux, 2013
				\item
					1st algo. of complexity \alert{quasi-poly}:\\
					Barbulescu, 2014 % change of name
			\end{itemize}
		\end{frame}
%%% ----------------------------------------------------------------------------------- %%%
	\subsection{From Sub-exp to Quasi-poly}

		\begin{frame}{Smooth - Set A Bound}
			\begin{definition}
				Given $B>0$. 
				$\forall n\in\mathbb{Z}$ is called \alert{$B$-smooth} if all its prime factors are no larger than $B$. 
				Thus denote \alert{factor basis} as
				$$\mathcal{F}(B) = \{ p\in\mathbb{N}: \text{prime and } p\leq B\}$$
			\end{definition}
			\begin{definition}
				Given $\beta\in\mathbb{Z}_{>0}$.
				$\forall f[X]\in\mathbb{F}_q[X]$ is called \alert{$\beta$-smooth} if all its irreducible factors are of degree no higher than $\beta$.
				Thus denote respective \alert{factor basis} as
				$$\mathcal{F}_q(\beta) = \{ I_k\in\mathbb{F}_q[X]: \text{irre. monic and deg. } k\leq \beta\}$$
			\end{definition}
		\end{frame}

		\begin{frame}{Index Calculus Method - Framework of Following Algo.}
			\begin{itemize}
				\item Given: $h \in  \mathbb{F}_Q^\times = <g>$; Find: $\log_g h$
					\pause
				\item Process:
					\begin{enumerate}
						\item\textbf{Main Phase}
						\begin{itemize}
							\item<3-> Fix parameter $B$, thus $\mathcal{F}(B)$ is also given.
							\item<4-> \alert{Smoothness Selection}: random $x\in[1, Q - 2]$ s.t. $g^x$ is $B$-smooth:
								$$g^x = \prod_{p\in\mathcal{F}(B)} p^{v(p, x)}$$
							\item<5-> \alert{Relation Collection}: 
								take $\log_g\cdot$ on both sides and 	substitute $\log_g p$ by unknown variable $x_p$ 
								(denoted as $\log_g p \leftrightarrow x_p$):
								$$x \equiv \sum_{p\in\mathcal{F}(B)} v(p, x) x_p \mod Q - 1$$
							\item<6-> \alert{Linea Algebra}: Solve system of linear equations.
						\end{itemize}
					\item
						\textbf{Individual Logarithm}\\
						\uncover<7->{
						Find $y\in[1, Q-2]$ s.t. $g^y h$ is $B$-smooth.} 
						\uncover<8->{Then factorization $g^y h = \prod p^{v_0(p)}$ implies
						$$\log_g h \equiv \sum v_0(p) \log_g p - y \mod Q - 1$$.}
					\end{enumerate}
			\end{itemize}
		\end{frame}

		\begin{frame}{Coppersmith's Method}
			\begin{itemize}
				\item Given: $h \in \mathbb{F}_Q^\times = <g>$ where $Q = q^k$; Find: $\log_g h$.
					\pause
				\item Process:
					\begin{enumerate}
						\item \textbf{Main Phase}
							\begin{itemize}
								\item<3-> Fix parameter $\beta$ then obtain $\mathcal{F}_q(\beta) = \{F(X):\cdots\}$. 
									Find $n$ satisfying $q^{n-1} < k \leq q^n$, fix $n_1 + n_2 = n$. 								
								\item<4-> Find $S(X)$ of deg.$\leq\beta$ s.t. $\exists I_k(X)|X^{q^n} - S(X)$ with a root $\alpha$.
								\item<5-> \alert{Smoothness Selection}: random $A(X)$, $B(X)$ of deg.$\leq\beta$ s.t. 
									both $C(X) = A(X) + X^{q^{n_1}}B(X)$ and $D(X) = C(X)^{q^{n_2}}$ %= A(X^{q^{n_2}}) + S(X)B(X^{q^{n_2}})$ 
									are $\beta$-smooth, notice deg.$\leq q^{n_1} + \beta$ and $(q^{n_2} + 1)\beta$ respectively.
								\item<6-> \alert{Relation Collection}: known
									$$(\prod F(X)^{v(C, F)})^{q^{n_2}} \equiv \prod F(X)^{v(D, F)} \mod I_k(X)$$
									\pause
									$X \leftrightarrow \alpha$, take $\log_g\cdot$ on both sides, and $\log_g F(\alpha)\leftrightarrow x_F$
									$$\sum(q^{n_2} v(C, F) - v(D, F)) x_F \equiv 0 \mod Q - 1$$
								\item<7-> \alert{Linear Algebra}: Solve system of linear equations.
							\end{itemize}
						\item \textbf{Individual Logarithm} \uncover<8->{Descent Method.}
					\end{enumerate}
			\end{itemize}
		\end{frame}
		
		\begin{frame}{NFS}
		\end{frame}
		
		\begin{frame}{Joux 13 and Barbulescu 14}
		\end{frame}

\section{Research Plan}


\end{document}
