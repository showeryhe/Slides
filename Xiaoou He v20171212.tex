\documentclass{beamer}

\mode<presentation>
{
	\usetheme{metropolis}
}

\usepackage[english]{babel}

\usepackage[utf8]{inputenc}

\usepackage{times}
\usepackage[T1]{fontenc}

\usepackage{amssymb, amsmath, amscd}
\usepackage{mathrsfs, amsfonts}

\usecolortheme{rose}
%\usepackage{color}
%\usepackage[usenames,dvipsnames]{color}
\bibliography{dlp v20171212}

\title[Short Paper Title]
{A Bird's Eye View\\on Discrete Logarithm Problems\\over Finite Fields}

\author[He]
{Xiao'ou He}

\institute[Universities of Somewhere and Elsewhere]
{Key Laboratory of Mathematics Mechanization, AMSS}

\date[Annual 2017]
{December 14, 2017}

\AtBeginSubsection[]
{
  \begin{frame}<beamer>{Outline}
    \tableofcontents[currentsection,currentsubsection]
  \end{frame}
}

\begin{document}

\begin{frame}
  \titlepage
\end{frame}

\begin{frame}{Outline}
  \tableofcontents[pausesections]
\end{frame}

%%% =================================================================================== %%%
\section{Motivation}

	\subsection{Origin of The Problem in Cryptography}
		\begin{frame}{Public-Key Cryptography}
			\begin{itemize}
				\item
					Proposed by Diffie and Hellman, 1976%\cite{DH76}
				\item
					Based on number-theoretical hard problems
					\begin{itemize}
						\item
							Integer Factorizations
						\item
							\alert{Discrete Logarithms}\\
							Concerning with two class of groups:
							\begin{enumerate}
								\item
									Elliptic curves
								\item
									\alert{Multiplicative groups of finite fields}
							\end{enumerate}
					\end{itemize}
			\end{itemize}
		\end{frame}
%%% ----------------------------------------------------------------------------------- %%%
	\subsection{Exposition of The Problem}
		% I suppose that the vocabulary "exposition" seems more high-end than the vougor "description"
		\begin{frame}{Case in General}
			\begin{itemize}
				\item
					Cyclic group: $(G, \cdot) = \langle g\rangle$.
				\item
					Exponential v. Logarithm
					\begin{itemize}
						\item
							$x \mapsto h = g^x$
						\item
							$h \mapsto x = \log_g{h}$
					\end{itemize}
				\item
					Generic Algorithms:\\
					\begin{itemize}
						\item
							Pohlig-Hellman: Two reductions - CRT and $p^e$ to $p$
						\item
							Collision Making: Baby-step giant-step, Pollard's rho Method, etc.
					\end{itemize}
			\end{itemize}
		\end{frame}
		
		\begin{frame}{Case over Finite Fields}
			\begin{itemize}
				\item
					Given: $Q$, $g$ and $h$, where $\langle g\rangle = \mathbb{F}_Q^\times$ containing $h$.
				\item
					Find: $\log_gh$.
			\end{itemize}
		\end{frame}
%%% =================================================================================== %%%
\section{Previous Work}
%%% ----------------------------------------------------------------------------------- %%%
	\subsection{Preliminaries}
		\begin{frame}{Complexity}
			\begin{block}{The $\mathcal{L}$ Notation}
				$$\mathcal{L}_Q(\beta, c) = \exp((c + o(1))(\log Q)^\beta (\log\log Q)^{1-\beta})$$
				\begin{itemize}
					\item 
						$c>0$ and $0 \le \beta \le 1$
					\item
						$\mathcal{L}_Q(\alert{0}, c) = (\log Q)^{c + o(1)} = \text{poly}(\log Q$),\\
						$\mathcal{L}_Q(\alert{1}, c) = (\exp(\log Q))^{c+o(1)} = \text{exp}(\log Q)$. 
					\item
						When $0<\beta<1$, $\mathcal{L}_Q(\beta, c)$ is \textbf{sub-exponential}.
				\end{itemize}
			\end{block}
			\begin{block}{Quasi-poly}
				$(\log Q)^{O(\log\log Q)}$ is \textbf{quasi-polynomial}, 
				which is smaller than any $\mathcal{L}_Q(\beta, \cdot)$ for $\beta >0$.
			\end{block}
		\end{frame}
%%% ----------------------------------------------------------------------------------- %%%	
	\subsection{From Sub-exp to Quasi-poly}		
		\begin{frame}{Overview}
			\begin{block}{$\mathcal{L}_Q(\alert{\frac{1}{2}}, \cdot)$ \textbf{Index Calculus Method}}
				Adleman, 1979 and Pohlig, 1977 independently
			\end{block}
			\begin{block}{$\mathcal{L}_Q(\alert{\frac{1}{3}}, \cdot)$ by \textbf{Coppersmith, 1984}}
				\begin{itemize}
						\item
							Originally $\mathbb{F}_{2^n}$
						\item
							More general: FFS by Adleman and Huang, 1999
						\item
							Medium and large char.: NFS by Gordon, 1993
				\end{itemize}
			\end{block}
			\begin{block}{$\mathcal{L}_Q(\alert{\frac{1}{4}}, \cdot)$ Joux, 2013 to \alert{quasi-poly} by BGJT, 2014}
				\begin{itemize}
					\item 
						For small char.: Take $\mathbb{F}_{2^n}$ for instace. Roughly $p \leq \mathcal{L}_Q(\frac{1}{3}, \cdot)$.
					\item
						Heuristics
				\end{itemize}
			\end{block}
		\end{frame}

		\begin{frame}{Smooth - Set A Bound}
			\begin{definition}
				Given $B>0$. 
				$\forall n\in\mathbb{Z}$ is called \alert{$B$-smooth} if all its prime factors are no larger than $B$. 
				Thus denote \alert{factor basis} as
				$$\mathcal{F}(B) = \{ p\in\mathbb{N}: \text{prime and } p\leq B\}$$
			\end{definition}
			\begin{definition}
				Given $\beta\in\mathbb{Z}_{>0}$.
				$\forall f[X]\in\mathbb{F}_q[X]$ is called \alert{$\beta$-smooth} if all its irreducible factors are of degree no higher than $\beta$.
				Thus denote respective \alert{factor basis} as
				$$\mathcal{F}_q(\beta) = \{F[X]\in\mathbb{F}_q[X]: \text{irre. monic and of deg.}\leq \beta\}$$
			\end{definition}
		\end{frame}

		\begin{frame}{Index Calculus Method - Framework of Following Algo.s}
			\begin{itemize}
				\item Given: $h \in  \mathbb{F}_Q^\times = \langle g\rangle$; Find: $\log_g h$
					\pause
				\item Process:
					\begin{enumerate}
						\item\textbf{Main Phase}
						\begin{itemize}
							\item<3-> Fix parameter $B$, thus $\mathcal{F}(B)$ is also given.
							\item<4-> \alert{Smoothness Selection}: random $x\in[1, Q - 2]$ s.t. $g^x$ is $B$-smooth:
								$$g^x = \prod_{p\in\mathcal{F}(B)} p^{v(p, x)}$$
							\item<5-> \alert{Relation Collection}: 
								take $\log_g\cdot$ on both sides and 	substitute $\log_g p$ by unknown variable $x_p$ 
								(denoted as $\log_g p \leftrightarrow x_p$):
								$$x \equiv \sum_{p\in\mathcal{F}(B)} v(p, x) x_p \mod Q - 1$$
							\item<6-> \alert{Linea Algebra}: Solve system of linear equations.
						\end{itemize}
					\item
						\textbf{Individual Logarithm}\\
						\uncover<7->{
						Find $y\in[1, Q-2]$ s.t. $g^y h$ is $B$-smooth.} 
						\uncover<8->{Then factorization $g^y h = \prod p^{v_0(p)}$ implies
						$$\log_g h \equiv \sum v_0(p) \log_g p - y \mod Q - 1$$.}
					\end{enumerate}
			\end{itemize}
		\end{frame}

		\begin{frame}{Coppersmith's Method}
			\begin{itemize}
				\item Given: $h \in \mathbb{F}_Q^\times = \langle g\rangle$ where $Q = q^k$; Find: $\log_g h$.
					\pause
				\item Process:
					\begin{enumerate}
						\item \textbf{Main Phase}
							\begin{itemize}
								\item<3-> Fix parameter $\beta$ then obtain $\mathcal{F}_q(\beta) = \{F(X):\cdots\}$. 
									Find $n$ satisfying $q^{n-1} < k \leq q^n$, fix $n_1 + n_2 = n$. 								
								\item<4-> Find $S(X)$ of deg.$\leq\beta$ s.t. $\exists I_k(X)|X^{q^n} - S(X)$ with a root $\alpha$.
								\item<5-> \alert{Smoothness Selection}: random $A(X)$, $B(X)$ of deg.$\leq\beta$ s.t. 
									both $C(X) = A(X) + X^{q^{n_1}}B(X)$ and $D(X) = C(X)^{q^{n_2}}$ %= A(X^{q^{n_2}}) + S(X)B(X^{q^{n_2}})$ 
									are $\beta$-smooth, notice deg.$\leq q^{n_1} + \beta$ and $(q^{n_2} + 1)\beta$ respectively.
								\item<6-> \alert{Relation Collection}: known
									$$(\prod F(X)^{v(C, F)})^{q^{n_2}} \equiv \prod F(X)^{v(D, F)} \mod I_k(X)$$
									\pause
									$X \leftrightarrow \alpha$, take $\log_g\cdot$ on both sides, and $\log_g F(\alpha)\leftrightarrow x_F$
									$$\sum(q^{n_2} v(C, F) - v(D, F)) x_F \equiv 0 \mod Q - 1$$
								\item<7-> \alert{Linear Algebra}: Solve system of linear equations.
							\end{itemize}
						\item \textbf{Individual Logarithm} \uncover<8->{Descent Method.}
					\end{enumerate}
			\end{itemize}
		\end{frame}
		
		%\begin{frame}{NFS}
		%\end{frame}
		
		\begin{frame}{Joux, 2013}
		\end{frame}
		
		\begin{frame}{BGJT, 2014}
		\end{frame}
%%% ----------------------------------------------------------------------------------- %%%
	\subsection{Summary}
		\begin{frame}{Paradigm with Transitions}
			\begin{enumerate}
				\item \textbf{Main Phase}
				\begin{itemize}
					\item Initiation: field extension brings freedom in presentation
					\item Smoothness Selection: \\Randomly choose $\Rightarrow$ Sieve $\Rightarrow$ Generate
					\item Relation Collection: \\Understanding the factor basis and (virtual) logarithms
					\item Linear Algebra: Take advantage of the sparseness
				\end{itemize}
				\item \textbf{Individual Logarithm Phase}: Descent Strategy
			\end{enumerate}
		\end{frame}
%%% =================================================================================== %%%
\section{Research Plan}
	\begin{frame}{Future Work}
		\begin{itemize}
			\item In Theory
			\begin{itemize}
				\item \textbf{Heuristics}: From assumption to rigorous.
					\begin{itemize}
						\item CWZ, 2014: Three Heuristics
						\item Existence of $(S_0, S_1)$ s.t. $\exists I_k(X)|S_1(X)X^q - S_0(X)$.
					\end{itemize}
				\item \textbf{Construction of the finite fields}:
					$\mathbb{F}_q[X]/\langle I_k[X]\rangle \Rightarrow \mathcal{O}_\mathbb{K}/P \Rightarrow$ ?
				\item \textbf{Relation Obtaining}:\\
					Balance two medium-deg. poly.s $\Rightarrow$ half-relation $\Rightarrow$ ? 
				\item \textbf{Tiny Goal}:\\
					Generalize small char. cases to medium and large ones? Truly poly. algo.?
			\end{itemize}
			\item In Practice: TBD.
		\end{itemize}
	\end{frame}
\end{document}
