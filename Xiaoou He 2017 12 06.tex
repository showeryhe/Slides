% !TEX TS-program = pdflatex
% !TEX encoding = UTF-8 Unicode

% This file is a template using the "beamer" package to create slides for a talk or presentation
% - Talk at a conference/colloquium.
% - Talk length is about 20min.
% - Style is ornate.

% MODIFIED by Jonathan Kew, 2008-07-06
% The header comments and encoding in this file were modified for inclusion with TeXworks.
% The content is otherwise unchanged from the original distributed with the beamer package.

\documentclass{beamer}


% Copyright 2004 by Till Tantau <tantau@users.sourceforge.net>.
%
% In principle, this file can be redistributed and/or modified under
% the terms of the GNU Public License, version 2.
%
% However, this file is supposed to be a template to be modified
% for your own needs. For this reason, if you use this file as a
% template and not specifically distribute it as part of a another
% package/program, I grant the extra permission to freely copy and
% modify this file as you see fit and even to delete this copyright
% notice. 


\mode<presentation>
{
  \usetheme{Metropolis}
  % or ...

  \setbeamercovered{transparent}
  % or whatever (possibly just delete it)
}


\usepackage[english]{babel}
% or whatever

\usepackage[utf8]{inputenc}
% or whatever

\usepackage{times}
\usepackage[T1]{fontenc}
% Or whatever. Note that the encoding and the font should match. If T1
% does not look nice, try deleting the line with the fontenc.
\usepackage{amssymb}

\title[Short Paper Title] % (optional, use only with long paper titles)
{A Bird's Eye View\\over Discrete Logarithm Problems\\in Finite Fields}

% \subtitle
% {Include Only If Paper Has a Subtitle}

\author[He] % (optional, use only with lots of authors)
{Xiao'ou He}

\institute[Universities of Somewhere and Elsewhere] % (optional, but mostly needed)
{
Key Laboratory of Mathematics Mechanization, AMSS}

\date[Annual 2017] % (optional, should be abbreviation of conference name)
{December 14, 2017}

% \subject{Theoretical Computer Science}
% This is only inserted into the PDF information catalog. Can be left
% out. 


% If you have a file called "university-logo-filename.xxx", where xxx
% is a graphic format that can be processed by latex or pdflatex,
% resp., then you can add a logo as follows:

% \pgfdeclareimage[height=0.5cm]{university-logo}{university-logo-filename}
% \logo{\pgfuseimage{university-logo}}



% Delete this, if you do not want the table of contents to pop up at
% the beginning of each subsection:
\AtBeginSubsection[]
{
  \begin{frame}<beamer>{Outline}
    \tableofcontents[currentsection,currentsubsection]
  \end{frame}
}


% If you wish to uncover everything in a step-wise fashion, uncomment
% the following command: 

%\beamerdefaultoverlayspecification{<+->}

\begin{document}

\begin{frame}
  \titlepage
\end{frame}

\begin{frame}{Outline}
  \tableofcontents
  % You might wish to add the option [pausesections]
\end{frame}


% Structuring a talk is a difficult task and the following structure
% may not be suitable. Here are some rules that apply for this
% solution: 

% - Exactly two or three sections (other than the summary).
% - At *most* three subsections per section.
% - Talk about 30s to 2min per frame. So there should be between about
%   15 and 30 frames, all told.

% - A conference audience is likely to know very little of what you
%   are going to talk about. So *simplify*!
% - In a 20min talk, getting the main ideas across is hard
%   enough. Leave out details, even if it means being less precise than
%   you think necessary.
% - If you omit details that are vital to the proof/implementation,
%   just say so once. Everybody will be happy with that.

\section{Motivation}

	\subsection{Origin of The Problem in Cryptography}
		\begin{frame}{Diffie-Hellman, 1976}
		  % - A title should summarize the slide in an understandable fashion
		  %   for anyone how does not follow everything on the slide itself.
	
			\begin{itemize}
				\item
				The basis of public key cryptography.
				\item
					One-way function.
				\item
					An example: discrete logarithm problem
			\end{itemize}
		\end{frame}

	\subsection{Exposition of The Problem}
		% I suppose that the vocabulary "exposition" seems more high-end than the vougor "description"
		\begin{frame}{Case in General}
			\begin{itemize}
				\item
					Cyclic group: $(G, \cdot) = <g>$.
				\item
					Exponential v. Logarithm
				\item
					Generic Algorithms:\\
					Pohlig-Hellman, Baby-step giant-step, Pollard's rho Method etc.		
			\end{itemize}
		\end{frame}
		
		\begin{frame}{Case in Finite Fields}
			\begin{itemize}
				\item
					Given: finite filed $\mathbb{F}_Q$ where $Q$ is power of prime $p$,
					a generator $g$ of $\mathbb{F}_Q^\times$ and arbitrary element $h\in\mathbb{F}_Q^\times$.
				\item
					Find: $\log_gh$.
			\end{itemize}
		\end{frame}

\section{Previous Work}

	\subsection{Preliminaries}
		\begin{frame}{The $\mathcal{L}$ Notation}
			$$\mathcal{L}_Q(\beta, c) = \exp((c + o(1))(\log Q)^\beta (\log\log Q)^{1-\beta})$$
			\begin{itemize}
				\item 
					$c>0$ and $0 \le \beta \le 1$
				\item
					$\mathcal{L}_Q(0, c) = (\log Q)^{c + o(1)} = \text{poly}(\log Q$),\\
					$\mathcal{L}_Q(1, c) = (\exp(\log Q))^{c+o(1)} = \text{exp}(\log Q)$. 
				\item
					When $0<\beta<1$, $\mathcal{L}_Q(\beta, c)$ is \textbf{sub-exponential}.
			\end{itemize}
		\end{frame}
	
	\subsection{Overview}		
		\begin{frame}{Overview}
			\begin{itemize}
				\item
					Index Calculus Method
						\begin{itemize}
							\item
								1st sub-exp algorithm, complexity $\mathcal{L}_Q(\frac{1}{2}, \cdot)$
							\item
								Adleman, 1979 and Pohlig 1977 independently
						\end{itemize}
				\item
					Coppersmith's Method
						\begin{itemize}
							\item
								1st algorithm of complexity $\mathcal{L}_Q(\frac{1}{3}, \cdot)$
							\item
								Originally for $Q$ power of 2
							\item
								Generalized to prime powers easily % what meaning
						\end{itemize}
			\end{itemize}
		\end{frame}
		
		\begin{frame}{Overview: Until 2013}
			All cases solved in $\mathcal{L}_Q(\frac{1}{3}, \cdot)$
			\begin{itemize}
				\item
					Small char: function field sieve (FFS)
				\item	
					Medium and large char: number field sieve (NFS)
			\end{itemize}
		\end{frame}

		\begin{frame}{Overview: Small Char}
			\begin{itemize}
				\item
					1st algorithm of complexity $\mathcal{L}_Q(\frac{1}{4}, \cdot)$: Frobenius representation, Joux 2013
				\item
					1st algorithm of complexity quasi-poly: Barbulescu, 2014
			\end{itemize}
		\end{frame}

	\subsection{From Sub-exp to Quasi-poly}

		\begin{frame}{Index Calculus Method}
		\end{frame}

		\begin{frame}{Coppersmith's Method}
		\end{frame}
		
		\begin{frame}{NFS}
		\end{frame}
		
		\begin{frame}{Joux 13 and Barbulescu 14}
		\end{frame}

\section{Research Plan}


\end{document}
